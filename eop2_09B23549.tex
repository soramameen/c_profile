% ファイル先頭から\begin{document}までの内容(プレアンブル)については,
% 基本的に { } の中を書き換えるだけでよい.
\documentclass[autodetect-engine,dvi=dvipdfmx,ja=standard,
               a4j,11pt]{bxjsarticle}

%%======== プレアンブル ============================================%%
% 用紙設定:指示があれば,適切な余白に設定しなおす
\RequirePackage{geometry}
\geometry{reset,paperwidth=210truemm,paperheight=297truemm}
\geometry{hmargin=25truemm,top=20truemm,bottom=25truemm,footskip=10truemm,headheight=0mm}
%\geometry{showframe} % 本文の"枠"を確認したければ,コメントアウト

% 設定:図の挿入
% http://www.edu.cs.okayama-u.ac.jp/info/tool_guide/tex.html#graphicx
\usepackage{graphicx}

% 設定:ソースコードの挿入
% http://www.edu.cs.okayama-u.ac.jp/info/tool_guide/tex.html#fancyvrb
\usepackage{fancyvrb}
\renewcommand{\theFancyVerbLine}{\texttt{\footnotesize{\arabic{FancyVerbLine}:}}}

%%======== レポートタイトル等 ======================================%%
% ToDo: 提出要領に従って,適切なタイトル・サブタイトルを設定する
\title{名簿管理プログラム \\
       |\Large{独自コマンドを添えて}|}

% ToDo: 自分自身の氏名と学生番号に書き換える
\author{氏名: 中嶋 空偉 (NAKAJIMA, Sorai) \\
        学生番号: 09B23549}

% ToDo: レポート課題等の指示に従って適切に書き換える
\date{出題日: 2024年6月12日 \\
      提出日: 2024年7月22日 \\
      締切日: 2024年7月24日 \\}  % 注:最後の\\は不要に見えるが必要.


%%======== 本文 ====================================================%%
\begin{document}
\maketitle
% 目次つきの表紙ページにする場合はコメントを外す
%{\footnotesize \tableofcontents \newpage}

%% 本文は以下に書く.課題に応じて適切な章立てを構成すること.
%% 章=\section,節=\subsection,項=\subsubsection である.

%--------------------------------------------------------------------%
\section{概要} \label{sec:abstract}
構造体と配列を用いて,大量のデータを扱うC言語プログラムの基本を学ぶ.
その過程で,標準入力から「ID, 氏名, 誕生日, 住所, 備考」からなる情報を管理する
名簿管理プログラムを作成する.このプログラムでは,上の形式で入力するとデータが保存される.
また,\verb|%|から始まるコマンドを入力すると,内容を表示,プログラムの停止,保存数の参照,探索,ファイルへの書き込み,
読み込み,データの整列などができる.

本レポートでは,探索,ファイルへの書き込み,読み込み,データの整列のコマンドを実装していく.


演習中に取り組んだ課題として,
以下の課題4から課題7についての内容を報告する.

\begin{description}
  \item[課題1]名簿管理プログラムの要件や仕様を踏まえたプログラム作成方針の検討
  \item[課題2]コマンド実装と考察(a)\%Fコマンド
  \item[課題3]コマンド実装と考察(b)\%Rコマンドと\%Wコマンド
  \item[課題4]コマンド実装と考察(c)\%Sコマンド
  \item[課題5]コマンド実装と考察(d)独自コマンド
  \item[課題6]プログラム全体に対する考察
  \item[課題7]発展課題の実装

  %本文の記述と矛盾しないように注意すること.
\end{description}

% 以下は,レポートB以降で必須の考察についての書き方の例.
また,考察課題として,以下の観点での考察をおこない,\ref{consider}章にまとめた.

\begin{description}
  \item[考察1]二分探索法の導入についての考察 (\ref{consider1}節)
  \item[考察2] メモリ上のデータ配置と構造体のサイズに関する考察(\ref{consider2}節)
  %本文の記述と矛盾しないように注意すること.
\end{description}
\clearpage
%--------------------------------------------------------------------%

\section{プログラムの作成方針} \label{sec:plan-of-programming}
    

本演習で作成したプログラムが満たすべき要件と仕様として,
「(1) 基本要件」と「(2) 基本仕様」を示す.
% 追加仕様も書くなら,書き変えを忘れずに

\subsection*{(1) 基本要件}
% ...ction* と,アスタリスクを入れている点に注意.数字無しの節や項目が作られる.

\begin{enumerate}
  \item プログラムは,その実行中,少なくとも10000件の名簿データをメモリ中に保持できるようにすること.
  \item 名簿データは,「ID,氏名,誕生日,住所,備考」を一つのデータとして扱えるようにすること.
  \item プログラムとしての動作や名簿データの管理のために,以下の機能を持つコマンドを実装すること.
  \begin{enumerate}
    \item プログラムの正常な終了
    \item 登録された名簿データのデータ数表示
    \item 名簿データの全数表示,および,部分表示
  \end{enumerate}
  \item 標準入力からのユーザ入力を通して,データ登録やデータ管理の操作を可能とすること.
  \item 標準出力にはコマンドの実行結果のみを出力すること.      % サンプルのため省略
\end{enumerate}
% Tips: enumerateの行間を調整するなら,以下の様に調整用のコマンドを入れる.
% \begin{enumerate}
%   \setlength{\parskip}{0em} \setlength{\itemsep}{0.25em}   % <--
%   \item ...
%   \item

\subsection*{(2) 基本仕様}

\begin{enumerate}
  \item 名簿データは,コンマ区切りの文字列(\textbf{CSV入力}と呼ぶ)で表されるものとし,%
        Listing~\ref{lst:csvdata}に示したようなテキストデータを処理できるようにする.%
        %CSV入力の詳細は,\ref{sec:sepc_csv}節に示す.  % <-- この1文は,レポートでは不要
  \item コマンドは,\%で始まる文字列(\textbf{コマンド入力}と呼ぶ)とし,表\ref{tab:commands}にあげたコマンドをすべて実装する
  \item 1つの名簿データは,C言語の構造体 (\texttt{struct}) を用いて,構造を持ったデータとしてプログラム中に定義し,利用する.
  \item 全名簿データは,"何らかのデータ構造"を用いて,メモリ中に保持できるようにする.
  \item コマンドの実行結果以外の出力は,標準エラー出力に出力する.  % サンプルのため省略
  
\end{enumerate}
\clearpage
\begin{table}[t]  %表の位置は原則として t または b.h や H は使わない.
   \centering 
    \caption{実装するコマンド}
    \label{tab:commands}
    \begin{tabular}{|l|ll|l|}
    \hline
      \textbf{コマンド} & \textbf{意味} &        & \textbf{備考} \\
    \hline\hline
      \verb|%Q|   & 終了                & (Quit)  & \\
      \hline
      \verb|%C|   & 登録件数の表示        & (C)  & \\
    \hline
      \verb|%P n| & 先頭から$n$件表示   & (Print) & $n$が$0$ $\to$ 全件表示,        \\
                  &                     &         & $n$が負 $\to$ 後ろから$-n$件表示 \\
                  \hline
   \verb|%R file|   & \verb|file| の読み込み            & (Read)  & \\
   \hline
   \verb|%W file|   & \verb|file| の書き出し& (Write)  &ファイルに上書きで書き出す. \\
   \hline
   \verb|%F word|   & 検索結果を表示                & (Find)  & \%P と同じ形式で表示\\
    \hline
    \verb|%S n|   & \verb|CSV|の\verb|n|番目の項目で整列                & (Sort)  & 表示はしない\\
    \hline
    \verb|%D id|   & プロファイルの削除.     & (delete)  &メモリの解放を行う\\
   \hline
    \end{tabular}
  \end{table}

\subsection{プログラムの全体像}
%practice7-1参照
\begin{enumerate}
  \item \textbf{標準入力から取得}

  標準入力から\texttt{get\_line}関数を用いて一行ずつ読み込む.
  \item \textbf{プロファイルデータに変換,もしくはコマンド処理}

  \texttt{parse\_line}関数によって最初の文字が\%のときはコマンド処理,
  そうでない場合はプロファイルデータとして処理を分ける.
  \item \textbf{プロファイルデータの作成}

  \texttt{new\_profile}関数によってプロファイルデータをつくる.
  データは\texttt{profile\_data\_store}配列に格納する.
  \item \textbf{コマンドの実行}

  \texttt{exec\_command}関数によってそれぞれのコマンドに引数が渡されて実行される.
  
\end{enumerate}
\subsection{名簿管理プログラムの実装方針}
\subsubsection{\%Fコマンド}
\begin{itemize}
  \item \textbf{機能} 引数として文字列を受け取る.引数とどれかの要素が一致するものを探索し,表示する.
  \item \textbf{実装方針} 引数と一致するかどうかをそれぞれの要素で\texttt{strcmp}関数を用いて判定する.
  一致した場合,表示する.この動きを\texttt{for}ループによってすべてのプロファイルで行う.
  \item \textbf{動作確認} 適当な数のプロファイルを与える.id,誕生日,名前,住所,コメントで検索して
  正しいものが出力されることを確認する.また,同じデータを持つものが複数あるときの動作も確認する.
  データにないものを入力して,見つからなかった時の動きを確認する,誕生日の入力形式についても確認する.
\end{itemize}
\subsubsection{\%Rコマンド}
\begin{itemize}
  \item \textbf{機能}引数としてファイル名を受け取る.一行ずつ読み取ってプロファイルデータとして読み込み.
  \item \textbf{実装方針} \texttt{fopen}関数によって読み取りモードでファイルを開く.
  \texttt{fgets}関数をファイル形式に対応させて一行ずつ読み込む.読み込んだデータは\texttt{new\_profile}関数で
  \texttt{profile}構造体形式にする.\texttt{profile\_data\_store}に格納していく.
  \item \textbf{動作確認}適当な数プロファイルデータが入った\texttt{csv}ファイルを用意する.
  \%Rコマンドによって読み込んで\%Pコマンドで\texttt{file}と同じ内容か確認する,
  同じ内容を二回読み込むことで上書きになっていないかを確認する.
\end{itemize}
\subsubsection{\%Wコマンド}
\begin{itemize}
  \item \textbf{機能}引数としてファイル名を受け取る.\texttt{profile\_data\_store}に格納されている\texttt{profile}を
  引数の\texttt{file}に書き込んでいく.
  \item \textbf{実装方針} \texttt{fopen}関数によって書き込みモードでファイルを開く.
  \texttt{profile\_data\_store}データを\texttt{fprintf}関数で\texttt{yyyy-mm-dd}形式に変換して
  \texttt{file}に書き込む.\texttt{for}ループによってすべてのプロファイルで行う.最後にはファイルをクローズする.
  \item \textbf{動作確認} 適当な数のプロファイルを与える.\texttt{file}に書き込んで同じ形式になっているか確認する.
  特に正しく改行されているか,データが入った\texttt{file}のとき上書きされるかも確認する,
\end{itemize}
\subsubsection{\%Sコマンド}
\begin{itemize}
  \item \textbf{機能} 引数としてint型の整数を受け取る.引数1,2,3,4,5それぞれでprofileのフィールドごとに
  並び替える.ソートアルゴリズムはバブルソートを採用する.
  \item \textbf{実装方針} \texttt{switch}文を用いてそれぞれの操作を分ける.
  
  比較用関数として
  \texttt{sort\_id,birthday,name...}を用意する.比較はすべて\texttt{sprintf}関数を用いて文字列に変換してから,
  \texttt{strcmp}関数を用いて行う.\texttt{swap\_profile}関数で並び替える.
  \item \textbf{動作確認} 適当な数のprofileデータを与える.それぞれのフィールドで正しくソートされるかについて確認する.
  また,1-5以外の引数の時の動きを確認する.異なるprofileデータが同じフィールドデータを持つときの
  動きも確認する.
\end{itemize}
\clearpage
\subsubsection{\%Dコマンド}
\begin{itemize}
  \item \textbf{機能} 引数としてint型の整数を受け取る.引数をidとして一致するidの
  profileを削除する.
  \item \textbf{実装方針} 1からidが一致するものを探索する.一致するものがあれば,
  そのprofileに次のprofileデータをコピーしていき,削除する.探索は,最後まで繰り返されるため,
  同じidが存在した場合,削除を繰り返す.削除する前にcommentフィールドのメモリの解放をする.
  \item \textbf{動作確認} 適当な数のプロファイルを与える.Dコマンドを実行して引数と同じものが削除されることを確認する.
  その際,メモリが解放されたかのメッセージを確認する.
  また,同じidのprofileを複数用意して,どのように動作するか確認する.一致するprofileがない時の動作も確認する.

\end{itemize}
\clearpage
%--------------------------------------------------------------------%
\section{プログラムの実装と考察}
  \subsection{コマンドの実装と考察(a)\texttt{\%F}コマンド}

    \subsubsection{プログラムの説明}
    
    この関数が呼び出されると、与えられた引数を使って\texttt{profile\_data\_store}配列内のプロファイルを検索する.

      \begin{itemize}
        \item \textbf{引数} 引数として文字列ポインタ\texttt{word}を受け取る.この\texttt{word}が探索する際のキーワードとなる.
        \item \textbf{動き}
        \begin{enumerate}
          \item \texttt{profile\_data\_store}配列を一つずつ見ていって、各プロファイルをチェックしていく.
          \item 各プロファイルに対して以下のチェックを行う.:
          \begin{enumerate}
            \item プロファイルIDを\texttt{sprintf}を使って文字列に変換し、\texttt{strcat}を使って\texttt{word}と比較する.IDが一致した場合、そのプロファイルの情報を出力する.
            \item \texttt{word}とプロファイルのコメント、名前、住所を上と同じように比較し、いずれかが一致した場合、そのプロファイルの情報を出力する.
            \item プロファイルの誕生日を文字列に変換し、\texttt{word}と同じように比較する.誕生日が一致した場合、そのプロファイルの情報を出力する.
          \end{enumerate}
        \end{enumerate}
      \end{itemize}
      \subsubsection{プログラムの動作確認}
      次のようなデータを読み込ませる.
      \begin{verbatim}
      5100046,The Bridge,1845-11-2,14 Seafield Road Longman Inverness,
      SEN Unit 2.0 Open
      5100127,Bower Primary School,1908-1-19,Bowermadden Bower Caithness,
      01955 641225 Primary 25 2.6 Open
      5100224,Canisbay Primary School,1928-7-5,
      Canisbay Wick,01955 611337 Primary 56 3.5 Open
      5100321,Castletown Primary School,1913-11-4,Castletown Thurso,
      01847 821256 01847 821256 Primary 137 8.5 Open
      5100429,Crossroads Primary School,1893-2-24,Dunnet Thurso,
      01847 851629 01847 851629 Primary 29 2.4 Open
      5100526,Dunbeath Primary School,1805-1-8,Dunbeath,
      01593 731286 Primary 13 1.1 Open
      \end{verbatim}
      \clearpage
      以下のようなコマンドを実行する.
      \begin{verbatim}
      %F 5100046
      %F Bower Primary School
      %F 1928-07-05
      %F Castletown Thurso
      %F 01847 851629 01847 851629 Primary 29 2.4 Open
      \end{verbatim}
      
      出力:
      \begin{verbatim}
      5100046,The Bridge,1845-11-2,14 Seafield Road Longman Inverness,
      SEN Unit 2.0 Open
      5100127,Bower Primary School,1908-1-19,Bowermadden Bower Caithness,
      01955 641225 Primary 25 2.6 Open
      5100224,Canisbay Primary School,1928-7-5,Canisbay Wick,
      01955 611337 Primary 56 3.5 Open
      5100321,Castletown Primary School,1913-11-4,Castletown Thurso,
      01847 821256 01847 821256 Primary 137 8.5 Open
      5100429,Crossroads Primary School,1893-2-24,Dunnet Thurso,
      01847 851629 01847 851629 Primary 29 2.4 Open
      5100526,Dunbeath Primary School,1805-1-8,Dunbeath,
      01593 731286 Primary 13 1.1 Open
      %F 5100046
      idが一致しました
      Id    : 5100046
      Name  : The Bridge
      Birth : 1845-11-02
      Addr. : 14 Seafield Road Longman Inverness
      Comm. : SEN Unit 2.0 Open
      
      %F Bower Primary School
      名前,住所,コメントのいづれかがヒットしました
      Id    : 5100127
      Name  : Bower Primary School
      Birth : 1908-01-19
      Addr. : Bowermadden Bower Caithness
      Comm. : 01955 641225 Primary 25 2.6 Open

      %F 1928-7-5
      %F 1928-07-05
      誕生日が一致しました
      Id    : 5100224
      Name  : Canisbay Primary School
      Birth : 1928-07-05
      Addr. : Canisbay Wick
      Comm. : 01955 611337 Primary 56 3.5 Open
      
      %F Castletown Thurso
      名前,住所,コメントのいづれかがヒットしました
      Id    : 5100321
      Name  : Castletown Primary School
      Birth : 1913-11-04
      Addr. : Castletown Thurso
      Comm. : 01847 821256 01847 821256 Primary 137 8.5 Open
      
      %F 01847 851629 01847 851629 Primary 29 2.4 Open
      名前,住所,コメントのいづれかがヒットしました
      Id    : 5100429
      Name  : Crossroads Primary School
      Birth : 1893-02-24
      Addr. : Dunnet Thurso
      Comm. : 01847 851629 01847 851629 Primary 29 2.4 Open
      \end{verbatim}
      それぞれのフィールドのデータに対して正確に探索できていることが確認できた.

    \subsubsection{実装にあたっての考察}
    この関数の実装にあたって\texttt{atoi}についてエラーが多く起こってしまったのでなぜうまくいかなかったのかについて考察する.
    \begin{itemize}
      \item \textbf{変換方法}:
        \texttt{atoi}は文字列が数字でない場合,いくつかの動作を行う.例えば,12345abcが入力されたとき12345と数字以外のものがくる
        まで変換して返す.abcが入力されたときは,変換できずに0が返される.数字の0が返されたのか,
        文字列を変換した結果返されたのがが分からないため,エラーの原因になりえる.
      \item \textbf{比較方法}:
        文字列と整数の比較
        を行う場合,変換がうまくいかなかった場合,その後に影響する場合もある.特に、文字列である場合、エラーが起こる.
    \end{itemize}
    いづれかの理由でエラーがおこってしまう.そのため今回は\texttt{sprintf}を使用して
    整数値を文字列に合わせるほうが確実だと考え,そちらを採用した.
\clearpage
  \subsection{コマンドの実装と考察(b)\texttt{\%R}コマンドと\texttt{\%W}コマンド}
    \subsubsection{\%Rコマンドの説明}
    この関数が呼び出されると,与えられたファイルからデータを読み取り,プロファイルを作成し,
    \texttt{profile\_data\_store}配列に格納する.
    \begin{itemize}
      \item\textbf{ファイルのオープン}:
      ファイルの名前が引数として渡されると,\texttt{fopen}関数により引数と一致するファイルが読み取りモード\texttt{"r"}で開かれる.
      開くのに失敗したときエラーとして出力される.
      \item\textbf{ファイルからのデータ読み込み}:
      \texttt{fget}関数を使って\texttt{file}から\texttt{line}に一行ずつ格納する.
      その際,\texttt{subst}関数で改行文字をnull文字に変える.不要な改行をなくす.
      \item \textbf{プロファイルの作成と格納}:
      \texttt{profile\_data\_item}が\texttt{MAX\_PROFILES}以上であればエラーを返す.それ未満であれば,
      \texttt{profile}型の構造体を定義し,profileに格納していく関数\texttt{new\_profile}に渡す.
      その際,\texttt{profile\_data\_item}を1増やす.作成できなかった場合はエラーを返す.
      \item \textbf{ファイルのクローズ}:
      データの読み込み後に\texttt{fclose}関数によってファイルを閉じる.


    \end{itemize}
    \subsubsection{\%Wコマンドの説明}
    この関数が呼び出されると,引数によって与えられたファイルに現在保持しているデータを書き込む,
    
    \begin{itemize}
      \item\textbf{ファイルのオープン}:
      ファイルの名前が引数として渡されると,\texttt{fopen}関数により引数と一致するファイルが書き込みモード\texttt{"r"}で開かれる.
      開くのに失敗したときエラーとして出力される.
      \item\textbf{ファイルへのデータ読み込み}:
      \texttt{profile\_data\_item}に基づいて/texttt{for}ループを実行する.
      それぞれのデータを構造体のポインタに格納する.その後,\texttt{csv}形式に合ったデータを
      \texttt{fprintf}関数によってファイルに保存していく.
      \item \textbf{ファイルのクローズ}:
      データの書き込み後に\texttt{fclose}関数によってファイルを閉じる.


    \end{itemize}
    \clearpage
    \subsubsection{\%Rプログラムの動作確認}
        \%Rコマンドによって読み込んで\%Pコマンドでファイルと同じ内容か確認する.同じファイルを二回読み込むことで、データが上書きされていないことを確認する.
        例えば、次のようなデータを含む \texttt{result\_R.csv} ファイルを用意する.
        \begin{verbatim}
        5100046,The Bridge,1845-11-2,14 Seafield Road Longman Inverness,
        SEN Unit 2.0 Open
        5100127,Bower Primary School,1908-1-19,Bowermadden Bower Caithness,
        01955 641225 Primary 25 2.6 Open
        5100224,Canisbay Primary School,1928-7-5,Canisbay Wick,
        01955 611337 Primary 56 3.5 Open
        5100321,Castletown Primary School,1913-11-4,Castletown Thurso,
        01847 821256 01847 821256 Primary 137 8.5 Open
        5100429,Crossroads Primary School,1893-2-24,Dunnet Thurso,
        01847 851629 01847 851629 Primary 29 2.4 Open
        \end{verbatim}

        以下のようなコマンドを実行する.
        \begin{verbatim}
        %R result_R.csv
        Data read from result_R.csv successfully.
        %P
        \end{verbatim}

        出力:
        \begin{verbatim}
        Id    : 5100046
        Name  : The Bridge
        Birth : 1845-11-02
        Addr. : 14 Seafield Road Longman Inverness
        Comm. : SEN Unit 2.0 Open

        Id    : 5100127
        Name  : Bower Primary School
        Birth : 1908-01-19
        Addr. : Bowermadden Bower Caithness
        Comm. : 01955 641225 Primary 25 2.6 Open

        Id    : 5100224
        Name  : Canisbay Primary School
        Birth : 1928-07-05
        Addr. : Canisbay Wick
        Comm. : 01955 611337 Primary 56 3.5 Open

        Id    : 5100321
        Name  : Castletown Primary School
        Birth : 1913-11-04
        Addr. : Castletown Thurso
        Comm. : 01847 821256 01847 821256 Primary 137 8.5 Open

        Id    : 5100429
        Name  : Crossroads Primary School
        Birth : 1893-02-24
        Addr. : Dunnet Thurso
        Comm. : 01847 851629 01847 851629 Primary 29 2.4 Open
        \end{verbatim}
        ファイルのデータがプロファイルデータとして読み込まれたことが確認できる.
        続けて同じファイルを再度読み込む.
        \begin{verbatim}
        %R result_R.csv
        Data read from result_R.csv successfully.
        %P
        \end{verbatim}

        出力:
        \begin{verbatim}
        Id    : 5100046
        Name  : The Bridge
        Birth : 1845-11-02
        Addr. : 14 Seafield Road Longman Inverness
        Comm. : SEN Unit 2.0 Open

        Id    : 5100127
        Name  : Bower Primary School
        Birth : 1908-01-19
        Addr. : Bowermadden Bower Caithness
        Comm. : 01955 641225 Primary 25 2.6 Open

        Id    : 5100224
        Name  : Canisbay Primary School
        Birth : 1928-07-05
        Addr. : Canisbay Wick
        Comm. : 01955 611337 Primary 56 3.5 Open

        Id    : 5100321
        Name  : Castletown Primary School
        Birth : 1913-11-04
        Addr. : Castletown Thurso
        Comm. : 01847 821256 01847 821256 Primary 137 8.5 Open

        Id    : 5100429
        Name  : Crossroads Primary School
        Birth : 1893-02-24
        Addr. : Dunnet Thurso
        Comm. : 01847 851629 01847 851629 Primary 29 2.4 Open

        Id    : 5100046
        Name  : The Bridge
        Birth : 1845-11-02
        Addr. : 14 Seafield Road Longman Inverness
        Comm. : SEN Unit 2.0 Open

        Id    : 5100127
        Name  : Bower Primary School
        Birth : 1908-01-19
        Addr. : Bowermadden Bower Caithness
        Comm. : 01955 641225 Primary 25 2.6 Open

        Id    : 5100224
        Name  : Canisbay Primary School
        Birth : 1928-07-05
        Addr. : Canisbay Wick
        Comm. : 01955 611337 Primary 56 3.5 Open

        Id    : 5100321
        Name  : Castletown Primary School
        Birth : 1913-11-04
        Addr. : Castletown Thurso
        Comm. : 01847 821256 01847 821256 Primary 137 8.5 Open

        Id    : 5100429
        Name  : Crossroads Primary School
        Birth : 1893-02-24
        Addr. : Dunnet Thurso
        Comm. : 01847 851629 01847 851629 Primary 29 2.4 Open
        \end{verbatim}
    データが上書きではなく追加になっていることが分かった.
    \subsubsection{\%Wプログラムの動作確認}
    まず,次のようなデータを入力する.
    \begin{verbatim}
      5100046,The Bridge,1845-11-2,14 Seafield Road Longman Inverness,
      SEN Unit 2.0 Open
      5100127,Bower Primary School,1908-1-19,Bowermadden Bower Caithness,
      01955 641225Primary 25 2.6 Open
      5100224,Canisbay Primary School,1928-7-5,Canisbay Wick,
      01955 611337 Primary 56 3.5 Open
      5100321,Castletown Primary School,1913-11-4,Castletown Thurso,
      01847 821256 01847 821256 Primary 137 8.5 Open
      5100429,Crossroads Primary School,1893-2-24,
      Dunnet Thurso,01847 851629 01847 851629 Primary 29 2.4 Open
    \end{verbatim}
    Wコマンドを入力する.
    \begin{verbatim}
      %W result_W.csv
    \end{verbatim}
    \texttt{result\_W.csv}を確認する.
    \begin{verbatim}
      5100046,The Bridge,1845-11-2,14 Seafield Road Longman Inverness,
      SEN Unit 2.0 Open
      5100127,Bower Primary School,1908-1-19,Bowermadden Bower Caithness,
      01955 641225 Primary 25 2.6 Open
      5100224,Canisbay Primary School,1928-7-5,Canisbay Wick,
      01955 611337 Primary 56 3.5 Open
      5100321,Castletown Primary School,1913-11-4,Castletown Thurso,
      01847 821256 01847 821256 Primary 137 8.5 Open
      5100429,Crossroads Primary School,1893-2-24,Dunnet Thurso,
      01847 851629 01847 851629 Primary 29 2.4 Open
    \end{verbatim}
    データが正しく書き出されている.
    さらに,次のようなデータがすでに入った\texttt{result\_W1}ファイルを用意する.
    \begin{verbatim}
      5100224,Canisbay Primary School,1928-07-05,Canisbay Wick,01955 611337 
      Primary 56 3.5 Open
      5100321,Castletown Primary School,1913-11-04,Castletown Thurso,
      01847 821256 01847 821256 Primary 137 8.5 Open
    \end{verbatim}
    次のようなデータを読み込ませる.
    \begin{verbatim}
      5100046,The Bridge,1845-11-02,14 Seafield Road Longman Inverness,
      SEN Unit 2.0 Open
      5100127,Bower Primary School,1908-01-19,Bowermadden Bower Caithness,
      01955 641225 Primary 25 2.6 Open
    \end{verbatim}
    Wコマンドを入力する.
    \begin{verbatim}
      %W result_W1.csv
    \end{verbatim}
    \texttt{result\_W1.csv}を確認すると
    \begin{verbatim}
      5100046,The Bridge,1845-11-02,14 Seafield Road Longman Inverness,
      SEN Unit 2.0 Open
      5100127,Bower Primary School,1908-01-19,Bowermadden Bower Caithness,
      01955 641225Primary 25 2.6 Open
    \end{verbatim}
    となって上書きされていることが分かる.
    

    \subsubsection{実装にあたっての考察}
    この関数の実装にあたって\texttt{get\_line}関数を変更するべきか,それとも新しくすべきか考えたので考察していく.
    \texttt{get\_line}関数を標準入力とファイルからの読み込みに対してどちらも対応した改良版のコードはこの後記載している.
    \begin{itemize}
      \item \textbf{まとめることのメリット}:
        \texttt{get\_line}関数をまとめることによって同じような関数を2回定義しなくてよいので簡潔なコードになる.
        また,エラーや修正が必要な改善が求められたときに修正箇所が少なくて済む.
      \item \textbf{まとめることのデメリット}:標準入力のときのみ追加したい事項や
      ファイルのときのみ追加したいエラーなどがあった場合に実装しづらい.また,\texttt{\%R}コマンドの時にのみしか必要ないにもかかわらず,
      多く使用されるであろう標準入力のときにも\texttt{if(file == NULL)}という条件分岐が発生するため,処理速度に影響があると推測できる.
      
    \end{itemize}
    \texttt{get\_line}関数をまとめることによって簡潔さなどが向上する反面,
    実際は時間のかかる処理ではないため,影響はないかもしれないが,1万という大きなデータを扱う上では大きな差になってしまうかもしれない.
    \clearpage
    \begin{Verbatim}[numbers=left, xleftmargin=10mm, numbersep=6pt,
      fontsize=\small, baselinestretch=0.8]
      int get_line(char *line, int size, FILE *file) {
    if (file == NULL) {
        if (fgets(line, 1024, stdin) != NULL) {
            subst(line, '\n', '\0');
            return 1;
        }
    } else {
        if (fgets(line, 1024, file) != NULL) {
            subst(line, '\n', '\0');
            return 1;
        }
    }
    return 0;
}
    \end{Verbatim}
  \subsection{コマンドの実装と考察(c)\texttt{\%S}コマンド}

  \subsubsection{プログラムの説明}
  この関数は、\texttt{cmd\_sort}が呼び出されると,int型の引数\texttt{column}によって
  id,名前,誕生日,住所,備考について並び替える.なお,並び替えた後の表示の機能はない.
  このソートはバブルソートのアルゴリズムに基づいている.

  \begin{itemize}
    \item \textbf{引数} 引数としてint型の引数\texttt{column}を受け取る.この\texttt{column}がソートする際のキーワードとなる.
    \item \textbf{動き}
    \begin{enumerate}
      \item \texttt{swap\_profile}関数:メモリ効率のために構造体のポインタを通して\texttt{profile}を交換する.
      \item \texttt{switch文}により\texttt{column}の値に応じてそれぞれソートを行う.
      \item バブルソートを行っていく,要素同士を比較して入れ替える関数をそれぞれの要素に対して用意している.
            
            \texttt{id}のとき,それぞれを\texttt{sprintf}によって文字列に変換する.\texttt{strcmp}により比較する.
            この値が正の時,\texttt{swap}関数によって入れ替える.

            \texttt{birthday}のとき,それぞれを\texttt{sprintf}によって年,月,日付をつなげて文字列に変換する.
            \texttt{strcmp}により比較する.この値が正の時,\texttt{swap}関数によって入れ替える.

            \texttt{name},\texttt{address},\texttt{comment}のとき,\texttt{strcmp}により比較する.この値が正の時,\texttt{swap}関数によって入れ替える.
  
  
    \end{enumerate}
  \end{itemize}
  

    \subsubsection{プログラムの動作確認}
    以下のようなプロファイルデータを読み込ませる.
    \begin{verbatim}
      5100127,Bower Primary School,1908-1-19,Bowermadden Bower Caithness,
      01955 641225 Primary 25 2.6 Open
      5100046,The Bridge,1845-11-2,14 Seafield Road Longman Inverness,
      SEN Unit 2.0 Open
      5100224,Canisbay Primary School,1928-7-5,Canisbay Wick,
      01955 611337 Primary 56 3.5 Open
      5100321,Castletown Primary School,1913-11-4,Castletown Thurso,
      01847 821256 01847 821256 Primary 137 8.5 Open
      5100429,Crossroads Primary School,1893-2-24,Dunnet Thurso,
      01847 851629 01847 851629 Primary 29 2.4 Open
    \end{verbatim}
    \%Sコマンドによってそれぞれのデータでソートする.
    \begin{verbatim}
      %S 1
      %P 5
      Id    : 5100046
      Name  : The Bridge
      Birth : 1845-11-02
      Addr. : 14 Seafield Road Longman Inverness
      Comm. : SEN Unit 2.0 Open

      Id    : 5100127
      Name  : Bower Primary School
      Birth : 1908-01-19
      Addr. : Bowermadden Bower Caithness
      Comm. : 01955 641225 Primary 25 2.6 Open

      Id    : 5100224
      Name  : Canisbay Primary School
      Birth : 1928-07-05
      Addr. : Canisbay Wick
      Comm. : 01955 611337 Primary 56 3.5 Open

      Id    : 5100321
      Name  : Castletown Primary School
      Birth : 1913-11-04
      Addr. : Castletown Thurso
      Comm. : 01847 821256 01847 821256 Primary 137 8.5 Open

      Id    : 5100429
      Name  : Crossroads Primary School
      Birth : 1893-02-24
      Addr. : Dunnet Thurso
      Comm. : 01847 851629 01847 851629 Primary 29 2.4 Open

      %S 2
      %P 5
      Id    : 5100127
      Name  : Bower Primary School
      Birth : 1908-01-19
      Addr. : Bowermadden Bower Caithness
      Comm. : 01955 641225 Primary 25 2.6 Open

      Id    : 5100224
      Name  : Canisbay Primary School
      Birth : 1928-07-05
      Addr. : Canisbay Wick
      Comm. : 01955 611337 Primary 56 3.5 Open

      Id    : 5100321
      Name  : Castletown Primary School
      Birth : 1913-11-04
      Addr. : Castletown Thurso
      Comm. : 01847 821256 01847 821256 Primary 137 8.5 Open

      Id    : 5100429
      Name  : Crossroads Primary School
      Birth : 1893-02-24
      Addr. : Dunnet Thurso
      Comm. : 01847 851629 01847 851629 Primary 29 2.4 Open

      Id    : 5100046
      Name  : The Bridge
      Birth : 1845-11-02
      Addr. : 14 Seafield Road Longman Inverness
      Comm. : SEN Unit 2.0 Open

      %S 3
      %P 5
      Id    : 5100046
      Name  : The Bridge
      Birth : 1845-11-02
      Addr. : 14 Seafield Road Longman Inverness
      Comm. : SEN Unit 2.0 Open

      Id    : 5100429
      Name  : Crossroads Primary School
      Birth : 1893-02-24
      Addr. : Dunnet Thurso
      Comm. : 01847 851629 01847 851629 Primary 29 2.4 Open

      Id    : 5100127
      Name  : Bower Primary School
      Birth : 1908-01-19
      Addr. : Bowermadden Bower Caithness
      Comm. : 01955 641225 Primary 25 2.6 Open

      Id    : 5100321
      Name  : Castletown Primary School
      Birth : 1913-11-04
      Addr. : Castletown Thurso
      Comm. : 01847 821256 01847 821256 Primary 137 8.5 Open

      Id    : 5100224
      Name  : Canisbay Primary School
      Birth : 1928-07-05
      Addr. : Canisbay Wick
      Comm. : 01955 611337 Primary 56 3.5 Open

      %S 4
      %P 5
      Id    : 5100046
      Name  : The Bridge
      Birth : 1845-11-02
      Addr. : 14 Seafield Road Longman Inverness
      Comm. : SEN Unit 2.0 Open

      Id    : 5100127
      Name  : Bower Primary School
      Birth : 1908-01-19
      Addr. : Bowermadden Bower Caithness
      Comm. : 01955 641225 Primary 25 2.6 Open

      Id    : 5100224
      Name  : Canisbay Primary School
      Birth : 1928-07-05
      Addr. : Canisbay Wick
      Comm. : 01955 611337 Primary 56 3.5 Open

      Id    : 5100321
      Name  : Castletown Primary School
      Birth : 1913-11-04
      Addr. : Castletown Thurso
      Comm. : 01847 821256 01847 821256 Primary 137 8.5 Open

      Id    : 5100429
      Name  : Crossroads Primary School
      Birth : 1893-02-24
      Addr. : Dunnet Thurso
      Comm. : 01847 851629 01847 851629 Primary 29 2.4 Open

      %S 5
      %P 5
      Id    : 5100321
      Name  : Castletown Primary School
      Birth : 1913-11-04
      Addr. : Castletown Thurso
      Comm. : 01847 821256 01847 821256 Primary 137 8.5 Open

      Id    : 5100429
      Name  : Crossroads Primary School
      Birth : 1893-02-24
      Addr. : Dunnet Thurso
      Comm. : 01847 851629 01847 851629 Primary 29 2.4 Open

      Id    : 5100224
      Name  : Canisbay Primary School
      Birth : 1928-07-05
      Addr. : Canisbay Wick
      Comm. : 01955 611337 Primary 56 3.5 Open

      Id    : 5100127
      Name  : Bower Primary School
      Birth : 1908-01-19
      Addr. : Bowermadden Bower Caithness
      Comm. : 01955 641225 Primary 25 2.6 Open

      Id    : 5100046
      Name  : The Bridge
      Birth : 1845-11-02
      Addr. : 14 Seafield Road Longman Inverness
      Comm. : SEN Unit 2.0 Open
    \end{verbatim}
      各データで問題なくソートされている.
      同じデータを持つものが複数あるときの動作も確認する.
      以下のようなプロファイルデータを読み込ませる.
    \begin{verbatim}
      5100127,Bower Primary School,1908-1-19,Bowermadden Bower Caithness,
      01955 641225 Primary 25 2.6 Open
      5100046,The Bridge,1845-11-2,14 Seafield Road Longman Inverness,
      SEN Unit 2.0 Open
      5100046,Canisbay Primary School,1928-7-5,Canisbay Wick,
      01955 611337 Primary 56 3.5 Open
      5100321,Castletown Primary School,1913-11-4,Castletown Thurso,
      01847 821256 01847 821256 Primary 137 8.5 Open
      5100046,Crossroads Primary School,1893-2-24,Dunnet Thurso,
      01847 851629 01847 851629 Primary 29 2.4 Open
    \end{verbatim}
    同じidを持つ場合以下のような結果になる.
    \begin{verbatim}
    %S 1
    %P
    Id    : 5100046
    Name  : The Bridge
    Birth : 1845-11-02
    Addr. : 14 Seafield Road Longman Inverness
    Comm. : SEN Unit 2.0 Open

    Id    : 5100046
    Name  : Canisbay Primary School
    Birth : 1928-07-05
    Addr. : Canisbay Wick
    Comm. : 01955 611337 Primary 56 3.5 Open

    Id    : 5100046
    Name  : Crossroads Primary School
    Birth : 1893-02-24
    Addr. : Dunnet Thurso
    Comm. : 01847 851629 01847 851629 Primary 29 2.4 Open

    Id    : 5100127
    Name  : Bower Primary School
    Birth : 1908-01-19
    Addr. : Bowermadden Bower Caithness
    Comm. : 01955 641225 Primary 25 2.6 Open

    Id    : 5100321
    Name  : Castletown Primary School
    Birth : 1913-11-04
    Addr. : Castletown Thurso
    Comm. : 01847 821256 01847 821256 Primary 137 8.5 Open
    \end{verbatim}
    同じidのとき,最初にデータが入っていた順番になる.
    \subsubsection{実装にあたっての考察}
    この関数の実装にあたって計算量がどのようになっているのか気になったので各関数の計算量,改善案について考察する.
    \begin{itemize}
      \item \textbf{計算量}:
        \texttt{swap\_profile}関数は構造体のコピーが3回ある.コピー操作はO(1)なので全体はO(1).
        \texttt{sort\_id}関数などの比較して並び替える関数は与えられたデータを文字列に変換して比較し,
        必要であれば\texttt{swap\_profile}関数を呼び出す.
        それぞれの操作はO(1)なので全体の操作はO(1).\texttt{cmd\_sort関数}はそれぞれの分岐に対して\texttt{for}ループにより
        N回ループの中でN回ループが実行される.その中で計算量$O(1)$の関数が実行される.よって全体の計算量は$O(N^2)$となる
      \item \textbf{改善案1}:クイックソートやマージソートなどのソート法に変える.
      これらのソート法は平均計算量が$N \log N$なので計算量を減らすことができる.
      \item \textbf{改善案2}:比較する際の文字列の変換を減らす.
      現状,比較する際にすべてのデータを文字列に変換しているが,idやbirthdayは数値のまま比較することで
      計算量は変換するために必要な分だけ計算量は減る.
        
    \end{itemize}
    

  \subsection{コマンドの実装と考察(d)独自コマンド}
    \subsubsection{\%Dコマンドの説明}
    この関数が呼び出されると,\texttt{id}が一致するものを探し,その\texttt{profile}データを削除する.
    \begin{itemize}
      \item\textbf{探索}:
      idが引数として渡されると,引数と一致するidをもつプロファイルが1から順に探索される.
      \item\textbf{メモリの解放}:
      \texttt{free}関数を用いて,探索で見つかったプロファイルの\texttt{comment}フィールドのメモリを解放する.
      メモリの解放がないとのちにメモリの圧迫により,エラーや動作が重くなる原因になる.
      \item \textbf{profileの削除}:
      見つかったプロファイルがあった場所にその次のプロファイルを格納していく.\texttt{for}ループによって一つずつずらして
      プロファイルを削除する.最後に\texttt{profile\_data\_items}を1減らす.

    \end{itemize}
    \subsubsection{プログラムの動作確認}
    
    \begin{verbatim}
      5100046,The Bridge,1845-11-2,14 Seafield Road Longman Inverness,
      SEN Unit 2.0 Open
      5100127,Bower Primary School,1908-1-19,Bowermadden Bower Caithness,
      01955 641225 Primary 25 2.6 Open
      5100224,Canisbay Primary School,1928-7-5,Canisbay Wick,
      01955 611337 Primary 56 3.5 Open
      5100321,Castletown Primary School,1913-11-4,Castletown Thurso,
      01847 821256 01847 821256 Primary 137 8.5 Open
      5100429,Crossroads Primary School,1893-2-24,Dunnet Thurso,
      01847 851629 01847 851629 Primary 29 2.4 Open
    \end{verbatim}
    \begin{verbatim}
      %D 5100127
      Memory was freed successfully
      %P
      Id    : 5100046
      Name  : The Bridge
      Birth : 1845-11-02
      Addr. : 14 Seafield Road Longman Inverness
      Comm. : SEN Unit 2.0 Open

      Id    : 5100224
      Name  : Canisbay Primary School
      Birth : 1928-07-05
      Addr. : Canisbay Wick
      Comm. : 01955 611337 Primary 56 3.5 Open

      Id    : 5100321
      Name  : Castletown Primary School
      Birth : 1913-11-04
      Addr. : Castletown Thurso
      Comm. : 01847 821256 01847 821256 Primary 137 8.5 Open

      Id    : 5100429
      Name  : Crossroads Primary School
      Birth : 1893-02-24
      Addr. : Dunnet Thurso
      Comm. : 01847 851629 01847 851629 Primary 29 2.4 Open

    \end{verbatim}
    メモリが解放されているメッセージが確認できる.
    idが一致しているプロファイルが削除されている.
    配列の一番後ろのデータを削除する.
    \begin{verbatim}
      %D 5100429
      Memory was freed successfully
      %P
      Id    : 5100046
      Name  : The Bridge
      Birth : 1845-11-02
      Addr. : 14 Seafield Road Longman Inverness
      Comm. : SEN Unit 2.0 Open

      Id    : 5100224
      Name  : Canisbay Primary School
      Birth : 1928-07-05
      Addr. : Canisbay Wick
      Comm. : 01955 611337 Primary 56 3.5 Open

      Id    : 5100321
      Name  : Castletown Primary School
      Birth : 1913-11-04
      Addr. : Castletown Thurso
      Comm. : 01847 821256 01847 821256 Primary 137 8.5 Open
    \end{verbatim}
    問題なく削除された.
    続いて,存在しないidを与える.
    \begin{verbatim}
      %D 000000
      %P
      Id    : 5100046
      Name  : The Bridge
      Birth : 1845-11-02
      Addr. : 14 Seafield Road Longman Inverness
      Comm. : SEN Unit 2.0 Open

      Id    : 5100224
      Name  : Canisbay Primary School
      Birth : 1928-07-05
      Addr. : Canisbay Wick
      Comm. : 01955 611337 Primary 56 3.5 Open

      Id    : 5100321
      Name  : Castletown Primary School
      Birth : 1913-11-04
      Addr. : Castletown Thurso
      Comm. : 01847 821256 01847 821256 Primary 137 8.5 Open

    \end{verbatim}
    そのままの状態のままプログラムは続いた.
    同じidのデータの2つを含めた3つのデータを読み込ませる.
    \begin{verbatim}
      5100046,The Bridge,1845-11-2,14 Seafield Road Longman Inverness,
      SEN Unit 2.0 Open
      5100046,The Bridge,1845-11-2,14 Seafield Road Longman Inverness,
      SEN Unit 2.0 Open
      5100321,Castletown Primary School,1913-11-4,Castletown Thurso,
      01847 821256 01847 821256 Primary 137 8.5 Open
    \end{verbatim}
    そのidを削除する,
    \begin{verbatim}
      %D 5100046
      Memory was freed successfully
      Memory was freed successfully
      %P
      Id    : 5100321
      Name  : Castletown Primary School
      Birth : 1913-11-04
      Addr. : Castletown Thurso
      Comm. : 01847 821256 01847 821256 Primary 137 8.5 Open
    \end{verbatim}
    同じidが複数ある場合,すべてのそのidをもつプロファイルは削除される.
    \subsubsection{実装にあたっての考察}
    このプログラムでは削除するものより後のデータをすべて移動させなければならない.
    ほかのデータ構造ついて考察する.
    \begin{itemize}
      \item \textbf{連結リスト}:
        連結リストを使用することにより,削除が効率的になる.削除するデータの前後のデータのみ
        変えればよいので今のプログラムより効率的である.さらに動的にサイズを変更できるのでメモリの面でも
        効率的である.しかし,すべてのプログラムを連結リスト用に
        変える必要があるので困難である.
      \item \textbf{ハッシュテーブル}:
        ハッシュテーブルは,非常に早く検索,削除ができ,メモリの面でも効率的である.
        しかし,順序を保持できないため今回のプログラムには向かない.
    \end{itemize}
%--------------------------------------------------------------------%
\section{プログラム全体の考察}\label{consider}
  \subsection{二分探索法の導入についての考察}\label{consider1}
  二分探索法は,ソートされたデータに対して効率的に検索を行うアルゴリズムである.これを導入することで得られるメリットとデメリットを挙げる.

    \textbf{メリット}
    \begin{itemize}
    \item \textbf{検索速度}:
    線形探索の時間計算量が$O(n)$であるのに対し,二分探索法は$O(\log n)$である.
    このため,データのサイズが大きくなるにつれて,二分探索法の検索速度の速さが目立ってくる.
    本プログラムでは1万近くのデータを管理するため,これは大きなメリットである.
    \item \textbf{シンプルな実装}:
    二分探索法はほかの検索に比べてシンプルなため実装しやすい.さらに,バグが起きたときに問題を突き止めやすい.
    \end{itemize}

    \textbf{デメリット}
    \begin{itemize}
    \item \textbf{ソートが必要}:
    二分探索法のためには,データがあらかじめソートされている必要がある.
    本プログラムのデータのソートには$O(N^2)$の時間がかかるため、
    データがソートされていないときとされているときの探索時間の差が非常に大きい.
    \item \textbf{複数のデータが見つけられない}:
    二分探索法では同じフィールドを持つプロファイルが複数あるとき,最初に探索されたもの
    しか探し出せない.これはそのものが見つかった時点でその数値を返す二分探索法の大きなデメリットである.
    線形探索では最初から最後のデータまで探し続けることができるので複数でも対応可能である.
    同じデータが存在しないと仮定できるidなどのみ使える.
    \item \textbf{動的データへの不適応}:
    二分探索法は.上にも述べたようにデータのソートが必要である.
    データが頻繁に追加・削除される本プログラムのような動的なデータにはあまり適していない.
    データが変更されるたびに再ソートが必要となり,これが非効率的である.
    \end{itemize}
    以上のように、二分探索法は特定の条件下で非常に有用であるが,
    使用する際には条件を十分に考慮する必要がある.
\clearpage
\subsection{メモリ上のデータ配置と構造体のサイズに関する考察}\label{consider2}
\subsubsection{変数や構造体のサイズ}
\begin{table}[h]
  \centering
  \begin{tabular}{|c|c|}
      \hline
      \textbf{データ型} & \textbf{サイズ(バイト)} \\
      \hline
      char & 1 \\
      \hline
      int & 4 \\
      \hline
      double & 8 \\
      \hline
      struct profile & 168 \\
      \hline
      id & 4 \\
      \hline
      name & 70 \\
      \hline
      birthday & 12 \\
      \hline
      address & 70 \\
      \hline
      comment & 8 \\
      \hline
      date & 12 \\
      \hline
  \end{tabular}
  \caption{各データ型のサイズ}
  \label{tab:data_sizes}
\end{table}
本プログラムに主に使われるデータのメモリをc言語プログラムによって求めた.
\ref{code1}章に使用したプログラムを記載している.
それを上の表にまとめている.
注目すべきは,構造体のフィールドごとのメモリを足したものより,構造体全体の
メモリのほうが大きくなっている点である.
なぜ,このようになるのか考察する.
\subsubsection{メモリアライメント}
メモリアライメントとはCPUの都合によって,コンパイラが適当に境界調整(アライメント)を行い,
構造体にパディングを挿入することである.intやdoubleは4の倍数のアドレスに保存される.
なので,char型の後にintやdoubleが入るときは3バイト分のパディングが挿入されることになる.
実際に,確認してみる.
int型のフィールドを2つもつstruct aとchar型とint型のフィールドを一つづつもつstruct bを用意した.
本来ならばaはサイズが8バイト,bが5バイトになるはずだが,
上の情報が正しければ,aはサイズが8バイトで後者も8バイトになるはずである.結果は以下のとおりである.
\ref{code2}章に使用したプログラムを記載している.

\begin{verbatim}
  Size of a: 8 bytes
  Size of b: 8 bytes
\end{verbatim}
これより間違いなくパディングが挿入されていることが分かる.
なお,パディングが挿入される理由はCPUのメモリへのアクセスの高速化である.\cite{www:label}
%--------------------------------------------------------------------%
\section{発展課題:二分探索法(binary search)の導入}
発展課題として全体考察で述べた二分探索法の導入を行う.
\%Sコマンドに組み込むことで,ソートされている場合は二分探索法,
されていない場合は従来の方法も考えた.しかしながら,\%Sコマンドが毎回ソートを確認しなければ
ならなくなってしまうため,新しい\%Bコマンドとして実装していく.
本課題ではidのソート用としていく.
\subsection{実装方針}
\begin{itemize}
  \item \textbf{機能} 引数としてint型の整数を受け取る.引数とidが一致するprofileを探索し,表示する.
  \item \textbf{実装方針} 引数を受け取ると,二分探索法のため,ソートされているか確認する.
  確認すると,二分探索のアルゴリズムを開始する.見つかると\texttt{print\_profile}関数によって出力される.
  \item \textbf{動作確認} 適当な数のプロファイルを与える.ソートされているときと,されていない時の動作を確認する.
  されている場合,一致するものを出力するか確認する.
  \item \textbf{比較} \%Fコマンドと比較する.比較用にidのみの\%Fコマンド,ソートされているか確認をしない\%Bコマンド,
  それぞれ\%f,\%bとして比較用コマンドを作っていく.理論上,$n : \log(n)$の割合の実行速度になるはずである.
\end{itemize}
\subsection{コマンドの実装と考察(c)\texttt{\%B}コマンド}

  \subsubsection{プログラムの説明}
  実装した部分は\ref{code3}章に記載している.
      \begin{itemize}
        \item \textbf{引数} 引数としてint型の引数\texttt{id}を受け取る.この\texttt{id}が探索する際のキーワードとなる.
        \item \textbf{動き}
        \begin{enumerate}
          \item \textbf{ソート確認}:まず,\texttt{profile\_data\_store}のprofileがidで
          ソートされているか比較文を用いて確認する.ソートされていないなら,エラー文を出力する.
          \item \textbf{二分探索}:left,right,mid変数を用意する.その時にmidを計算しておく.
          midが目的のidと一致していればmidを返して終了.大きければmid+1をleftとして,小さければmid-1をrightとして
          繰り返す.最後まで見つからなかったとき-1を返す.その後見つからなかったことを出力する.
          -1以外の時はmidのprofileを\texttt{print\_profile}関数で出力して終了する.
        \end{enumerate}
      \end{itemize}
      

    \subsubsection{プログラムの動作確認}
    まず,ソートされたデータを与える.
    \begin{verbatim}
      1,The Bridge,1845-11-2,14 Seafield Road Longman Inverness,
      SEN Unit 2.0 Open
      2,Bower Primary School,1908-1-19,Bowermadden Bower Caithness,
      01955 641225 Primary 25 2.6 Open
      3,Canisbay Primary School,1928-7-5,Canisbay Wick,
      01955 611337 Primary 56 3.5 Open
    \end{verbatim}
    1と4を探索する.
    \begin{verbatim}
      %B 1
      binarysearchを行います
      二分探索法によってこのidを持つものが見つかりました
      ID: 1
      Name: The Bridge
      Birthday: 1845-11-2
      Address: 14 Seafield Road Longman Inverness
      Comment: SEN Unit 2.0 Open

      %B 4
      binarysearchを行います
      このidを持つprofileは存在しません
    \end{verbatim}
    上のように二分探索によって探索されたことが確認できる.
    存在しない場合は出力する.
    次にソートされていないデータを与える.
    \begin{verbatim}
      1,The Bridge,1845-11-2,14 Seafield Road Longman Inverness,
      SEN Unit 2.0 Open
      3,Bower Primary School,1908-1-19,Bowermadden Bower Caithness,
      01955 641225 Primary 25 2.6 Open
      2,Canisbay Primary School,1928-7-5,Canisbay Wick,
      01955 611337 Primary 56 3.5 Open
    \end{verbatim}
    \begin{verbatim}
      %B 1
      binarysearchを行います
      このデータはソートされていません
    \end{verbatim}
    上のようにソートされていないと出力する.
    \subsubsection{線形探索との比較}
    比較ではfindコマンドをidのみように変えたものと
    ソートされているか確認しない二分探索法のコマンドを
    用意した.それぞれ探索開始時間,終了時間から
    時間を計測できるようなプログラムになっている.
    実験用関数は\ref{code4}章に掲載している.
    \begin{table}[h]
      \centering
      \begin{tabular}{|c|c|c|c|c|c|}
      \hline
      インデックス & ID & 線形探索の実行時間 (\%f) & 二分探索の実行時間 (\%b) \\ \hline
      1番目 & 5100046 & 0.000002 秒 & 0.000036 秒  \\ \hline
      500番目 & 5250927 & 0.000077 秒 & 0.000003 秒   \\ \hline
      1000番目 & 5506220 & 0.000014 秒 & 0.000003 秒  \\ \hline
      2000番目 & 8242143 & 0.000022 秒 & 0.000023 秒  \\ \hline
      2886番目 & 8681139 & 0.000375 秒 & 0.000003 秒 \\ \hline
      \textbf{合計} & & \textbf{0.000490 秒} & \textbf{0.000068 秒}  \\ \hline
      \end{tabular}
      \caption{線形探索と二分探索の実行時間比較}
      \end{table}
  ソートされたものという限定条件があるものの合計時間にはかなりの違いがみられた.
  それぞれ見てみると,線形探索は初めのほうは実行時間がかなり速いが,最後には約200倍
  なっている.それ比べて二分探索は安定して早いという結果になった.
  \clearpage
%--------------------------------------------------------------------%
% \section{感想}


%--------------------------------------------------------------------%
\section{作成したプログラムのソースコード}
% 参照を適切に直すか,記述を見直す必要がある.

作成したプログラムを以下に添付する.
なお,\ref{sec:abstract}章に示した課題については,
% \ref{xxxx}章で示したようにすべて正常に動作したことを付記しておく.

% Verbatim environment
% プリアンブルで \usepackage{fancyvrb} が必要.
%   - numbers           行番号を表示.left なら左に表示.
%   - xleftmargin       枠の左の余白.行番号表示用に余白を与えたい.
%   - numbersep         行番号と枠の間隙 (gap).デフォルトは 12 pt.
%   - fontsize          フォントサイズ指定
%   - baselinestretch   行間の大きさを比率で指定.デフォルトは 1.0.
\begin{Verbatim}[numbers=left, xleftmargin=10mm, numbersep=6pt,
                    fontsize=\small, baselinestretch=0.8]
#include <stdio.h>
#include <stdlib.h>
#include <string.h>

#define MAX_STR_LEN  69
#define MAX 1024
#define MAX_PROFILES 10000

int subst(char *str, char c1, char c2) {
    int count = 0;
    while (1) {
        if (*str == c1) {
            *str = c2;
            count++;
        }
        str++;
        if (*str == '\0') {
            return count;
        }
    }
}

int get_line(char *line) {
    if (fgets(line, 1024, stdin) != NULL) {
        subst(line, '\n', '\0');
        return 1;
    } else {
        return 0;
    }
}

int split(char *str, char *ret[], char sep, int max) {
    int i = 0;
    ret[i] = str;
    while (*str ) {
        if (*str == sep) {
            *str = '\0';
            ret[++i] = str + 1;
            
        } else if (*str == '\0') {
            return i + 1;
        }
        str++;
    }
    return i + 1;
}

struct date {
    int y;
    int m;
    int d;
};

struct profile {
    int id;
    char name[MAX_STR_LEN + 1];
    struct date birthday;
    char address[MAX_STR_LEN + 1];
    char *comment;
};

struct date *new_date(struct date *d, char *str) {
    char *ptr[3];

    if (split(str, ptr, '-', 3) != 3)
        return NULL;

    d->y = atoi(ptr[0]);
    d->m = atoi(ptr[1]);
    d->d = atoi(ptr[2]);

    return d;
}

struct profile *new_profile(struct profile *p, char *csv) {
    char *ptr[5];

    if (split(csv, ptr, ',', 5) != 5 )
        return NULL;

    p->id = atoi(ptr[0]);
    strncpy(p->name, ptr[1], MAX_STR_LEN);
    p->name[MAX_STR_LEN] = '\0';

    if (new_date(&p->birthday, ptr[2]) == NULL)
        return NULL;

    strncpy(p->address, ptr[3], MAX_STR_LEN);
    p->address[MAX_STR_LEN] = '\0';

    p->comment = (char *)malloc(sizeof(char) * (strlen(ptr[4]) + 1));
    if (p->comment == NULL) {
    fprintf(stderr, "Memory allocation failed for comment.\n");
    return NULL;
}
    strcpy(p->comment, ptr[4]);

    return p;
}

struct profile profile_data_store[MAX_PROFILES];
int profile_data_nitems = 0;
void swap_profile(struct profile *p1, struct profile *p2)
{
    struct profile tmp;

    tmp = *p2;
    *p2 = *p1;
    *p1 = tmp;
}
void print_profile(struct profile p) {
    printf("ID: %d\n", p.id);
    printf("Name: %s\n", p.name);
    printf("Birthday: %d-%d-%d\n", p.birthday.y, p.birthday.m, p.birthday.d);
    printf("Address: %s\n", p.address);
    printf("Comment: %s\n", p.comment);
}
void sort_id(struct profile *p,struct profile *q){
    char str_p[MAX] ="";
    char str_q[MAX]="";
    sprintf(str_p,"%d",p->id);
    sprintf(str_q,"%d",q->id);
    if (strcmp(str_p,str_q)>0){
        swap_profile(p,q);
    }
}
void sort_birthday(struct profile *p,struct profile *q){
    char str_p[MAX] ="";
    char str_q[MAX]="";
    sprintf(str_p,"%04d,%02d,%02d",p->birthday.y,p->birthday.m,p->birthday.d);
    sprintf(str_q,"%04d,%02d,%02d",q->birthday.y,q->birthday.m,q->birthday.d);
    
    if (strcmp(str_p,str_q)>0){
        swap_profile(p,q);
    }
}
void sort_name(struct profile *p,struct profile *q){
    char str_p[MAX] ="";
    char str_q[MAX]="";
    sprintf(str_p,"%s",p->name);
    sprintf(str_q,"%s",q->name);
    if (strcmp(str_p,str_q)>0){
        swap_profile(p,q);
    }
}
void sort_add(struct profile *p,struct profile *q){
    char str_p[MAX] ="";
    char str_q[MAX]="";
    sprintf(str_p,"%s",p->address);
    sprintf(str_q,"%s",q->address);
    if (strcmp(str_p,str_q)>0){
        swap_profile(p,q);
    }
}
void sort_comm(struct profile *p,struct profile *q){
    char str_p[MAX] ="";
    char str_q[MAX]="";
    sprintf(str_p,"%s",p->comment);
    sprintf(str_q,"%s",q->comment);
    if (strcmp(str_p,str_q)>0){
        swap_profile(p,q);
    }
}


void cmd_quit() {
    exit(0);
}

void cmd_check() {
    printf("%d profile(s)\n", profile_data_nitems);
    
}

void cmd_print(int count) {
    int j = 0;

    if (count == 0) {
        count = profile_data_nitems;
    }
    if (count < 0) {
        count = -count;
        if (count > profile_data_nitems) {
            count = profile_data_nitems;
        }
        for (j = profile_data_nitems - count; j < profile_data_nitems; ++j) {
            printf("Id    : %d\n", profile_data_store[j].id);
            printf("Name  : %s\n", profile_data_store[j].name);
            printf("Birth : %04d-%02d-%02d\n", profile_data_store[j].birthday.y, profile_data_store[j].birthday.m, profile_data_store[j].birthday.d);
            printf("Addr. : %s\n", profile_data_store[j].address);
            printf("Comm. : %s\n\n", profile_data_store[j].comment);
        }
    } else {
        if (count > profile_data_nitems) {
            count = profile_data_nitems;
        }
        for (j = 0; j < count; ++j) {
            printf("Id    : %d\n", profile_data_store[j].id);
            printf("Name  : %s\n", profile_data_store[j].name);
            printf("Birth : %04d-%02d-%02d\n", profile_data_store[j].birthday.y, profile_data_store[j].birthday.m, profile_data_store[j].birthday.d);
            printf("Addr. : %s\n", profile_data_store[j].address);
            printf("Comm. : %s\n\n", profile_data_store[j].comment);
        }
    }
}



void cmd_read(char *filename) {
    // fprintf(stderr, "Read (%%R) command is invoked with arg: '%s'\n", filename);
    FILE *file = fopen(filename, "r");
    if (file == NULL) {
        perror("Failed to open file for reading");
        return;
    }

    char line[MAX + 1];
    while (fgets(line, sizeof(line), file)) {
        subst(line, '\n', '\0');
        if (profile_data_nitems < MAX_PROFILES) {
            struct profile new_profile_data;
            if (new_profile(&new_profile_data, line) != NULL) {
                profile_data_store[profile_data_nitems] = new_profile_data;
                profile_data_nitems++;
            } else {
                fprintf(stderr, "Failed to create profile from line: %s\n", line);
            }
        } else {
            fprintf(stderr, "Maximum number of profiles reached.\n");
            break;
        }
    }

    fclose(file);
    printf("Data read from %s successfully.\n", filename);
}

void cmd_write(char *filename) {
    // fprintf(stderr, "Write (%%W) command is invoked with arg: '%s'\n", filename);
    FILE *file = fopen(filename, "w");
    if (file == NULL) {
        perror("Failed to open file for writing");
        return;
    }

    for (int i = 0; i < profile_data_nitems; i++) {
        struct profile *p = &profile_data_store[i];
        fprintf(file, "%d,%s,%04d-%02d-%02d,%s,%s\n",
                p->id,
                p->name,
                p->birthday.y,
                p->birthday.m,
                p->birthday.d,
                p->address,
                p->comment);
    }

    fclose(file);
    printf("Data written to %s successfully.\n", filename);
}

void cmd_find(char *word) {
    // fprintf(stderr, "Find (%%F) command is invoked with arg: '%s'\n", word);
    int i;
    struct profile *p ;
    

    for(i=0; i<sizeof(profile_data_store)/sizeof(profile_data_store[0]); i++) {

        p = &profile_data_store[i];
        char s[20];
        sprintf(s, "%d", p->id);
            if (strcmp(s, word) == 0){
                
            printf("idが一致しました\n");
            printf("Id    : %d\n", profile_data_store[i].id);
            printf("Name  : %s\n", profile_data_store[i].name);
            printf("Birth : %04d-%02d-%02d\n", profile_data_store[i].birthday.y, profile_data_store[i].birthday.m, profile_data_store[i].birthday.d);
            printf("Addr. : %s\n", profile_data_store[i].address);
            printf("Comm. : %s\n\n", profile_data_store[i].comment);
            continue;
        }

        
        if (p->comment && (!strcmp(word, p->comment) || !strcmp(word, p->name) || !strcmp(word, p->address))) {
            printf("名前,住所,コメントのいづれかがヒットしました\n");
            printf("Id    : %d\n", profile_data_store[i].id);
            printf("Name  : %s\n", profile_data_store[i].name);
            printf("Birth : %04d-%02d-%02d\n", profile_data_store[i].birthday.y, profile_data_store[i].birthday.m, profile_data_store[i].birthday.d);
            printf("Addr. : %s\n", profile_data_store[i].address);
            printf("Comm. : %s\n\n", profile_data_store[i].comment);
            continue;
        }

        
        char birth[20];
    sprintf(birth, "%04d-%02d-%02d", p->birthday.y, p->birthday.m, p->birthday.d);
    if (!strcmp(word, birth)) {
        printf("誕生日が一致しました\n");
        printf("Id    : %d\n", p->id);
        printf("Name  : %s\n", p->name);
        printf("Birth : %04d-%02d-%02d\n", p->birthday.y, p->birthday.m, p->birthday.d);
        printf("Addr. : %s\n", p->address);
        printf("Comm. : %s\n\n", p->comment);
    }
    }
}

void cmd_sort(int column) {
    // fprintf(stderr, "Sort (%%S) command is invoked with arg: '%d'\n", column);
    int i = 0, j = 0;
 
    switch(column){

    case 1:
            for (i = 0; i < profile_data_nitems; i++) {
            for (j = 0; j < profile_data_nitems - 1; j++) {
                sort_id(&profile_data_store[j],&profile_data_store[j+1]);
                }
            }
            break;
    case 2:
        int i = 0, j = 0;
            for (i = 0; i < profile_data_nitems; i++) {
            for (j = 0; j < profile_data_nitems - 1; j++) {
                sort_name(&profile_data_store[j],&profile_data_store[j+1]);
                }
            }
            break;
    case 3:
        
            for (i = 0; i < profile_data_nitems; i++) {
            for (j = 0; j < profile_data_nitems - 1; j++) {
                sort_birthday(&profile_data_store[j],&profile_data_store[j+1]);
                }
            }
            break;
    case 4:
            for (i = 0; i < profile_data_nitems; i++) {
            for (j = 0; j < profile_data_nitems - 1; j++) {
                sort_add(&profile_data_store[j],&profile_data_store[j+1]);
                }
            }
            break;
    case 5:
            for (i = 0; i < profile_data_nitems; i++) {
            for (j = 0; j < profile_data_nitems - 1; j++) {
                sort_comm(&profile_data_store[j],&profile_data_store[j+1]);
                }
            }
        break;
    }
}
void cmd_delete(int id){
    int i=0, j=0;
    for (i = 0; i <profile_data_nitems; i++){
        if (profile_data_store[i].id == id){
            free(profile_data_store[i].comment);
            printf("Memory was freed successfully\n");
            for(j = i; j < profile_data_nitems; j++){
                profile_data_store[j] = profile_data_store[j+1];
            }
            profile_data_nitems--;
            i--;
        }

    }
   

}

void exec_command(char cmd, char *param) {
    switch (cmd) {
        case 'Q': cmd_quit(); break;
        case 'C': cmd_check(); break;
        case 'P': cmd_print(atoi(param)); break;
        case 'R': cmd_read(param); break;
        case 'W': cmd_write(param); break;
        case 'F': cmd_find(param); break;
        case 'S': cmd_sort(atoi(param)); break;
        case 'D': cmd_delete(atoi(param)); break;

        default:
            fprintf(stderr, "Invalid command %c: ignored.\n", cmd);
            break;
    }

}

void parse_line(char *line) {
    char cmd;
    char *param;

    if (line[0] == '%') {
        cmd = line[1];
        param = line + 3;
        exec_command(cmd, param);
    } else {
        if (profile_data_nitems < MAX_PROFILES) {
            struct profile new_profile_data;
            if (new_profile(&new_profile_data, line) != NULL) {
                profile_data_store[profile_data_nitems] = new_profile_data;
                profile_data_nitems++;
            } else {
                fprintf(stderr, "Failed to create profile.\n");
            }
        } else {
            fprintf(stderr, "Maximum number of profiles reached.\n");
        }
    }
}
void free_profiles() {/*メモリ解放用関数*/
    int i = 0;
    for (i = 0; i < profile_data_nitems; i++) {
        free(profile_data_store[i].comment);
    }
}
int main() {
    char line[MAX + 1];
    while (get_line(line)) {
        parse_line(line);
    }
    free_profiles();
    return 0;
}
                   
                    
\end{Verbatim}
\subsection{検証用コード1}\label{code1}
\begin{Verbatim}[numbers=left, xleftmargin=10mm, numbersep=6pt,
  fontsize=\small, baselinestretch=0.8]
  #include <stdio.h>

#define MAX_STR_LEN  69
#define MAX 1024
#define MAX_PROFILES 10000
struct date {
    int y;
    int m;
    int d;
};

struct profile {
    int id;
    char name[MAX_STR_LEN + 1];
    struct date birthday;
    char address[MAX_STR_LEN + 1];
    char *comment;
};

int main() {
    struct profile person;
    printf("Size of char: %lu bytes\n", sizeof(char));
    printf("Size of int: %lu bytes\n", sizeof(int));
    printf("Size of double: %lu bytes\n", sizeof(double));
    printf("Size of struct profile: %lu bytes\n", sizeof(struct profile));
    printf("Size of struct id: %lu bytes\n", sizeof(person.id));
    printf("Size of struct name: %lu bytes\n", sizeof(person.name));
    printf("Size of struct birthday: %lu bytes\n", sizeof(person.birthday));
    printf("Size of struct address: %lu bytes\n", sizeof(person.address));
    printf("Size of struct comment: %lu bytes\n", sizeof(person.comment));
    printf("Size of struct date: %lu bytes\n", sizeof(struct date));
    return 0;
}
\end{Verbatim}
\subsection{検証用コード2}\label{code2}
\begin{Verbatim}[numbers=left, xleftmargin=10mm, numbersep=6pt,
  fontsize=\small, baselinestretch=0.8]
  #include <stdio.h>

  struct a {
      int a;
      int b;
  };
  struct b {
      char a;
      int b;
  };
  int main() {
      printf("Size of a: %lu bytes\n", sizeof(struct a));
      printf("Size of b: %lu bytes\n", sizeof(struct b));
      return 0;
  }


\end{Verbatim}
\subsection{検証用コード3}\label{code3}
\begin{Verbatim}[numbers=left, xleftmargin=10mm, numbersep=6pt,
  fontsize=\small, baselinestretch=0.8]
  int binary_search(int id) {
    int left = 0;
    int right = profile_data_nitems - 1;

    while (left <= right) {
        int mid = left + (right - left) / 2;

     
        if (profile_data_store[mid].id == id) {
            return mid;
        }
     
        else if (profile_data_store[mid].id < id) {
            left = mid + 1;
        }
 
        else {
            right = mid - 1;
        }
    }

    return -1;
}


int is_sorted() {
    int i = 0;
    for (i = 0; i < profile_data_nitems-1; i++){
        if(profile_data_store[i].id > profile_data_store[i+1].id){
            printf("このデータはソートされていません\n");
            return 0;
        }
      
    }
    return 1;
}
void binary(int id){
    printf("binarysearchを行います\n");
    if (is_sorted()) {
        int result = 0;
        result = binary_search(id);
        if(result == -1) {
            printf("このidを持つprofileは存在しません\n");
        }
        else{
            printf("二分探索法によってこのidを持つものが見つかりました\n");
            print_profile(profile_data_store[result]);
        }
        
    }
}
\end{Verbatim}
\subsection{検証用コード4}\label{code4}
\begin{Verbatim}[numbers=left, xleftmargin=10mm, numbersep=6pt,
  fontsize=\small, baselinestretch=0.8]
  void find(int id){
    int i = 0;
    clock_t start, end;
    double cpu_time_used;
    start = clock();

    for(i = 0; i< profile_data_nitems; i++){
        if(profile_data_store[i].id == id){
            end = clock();
            cpu_time_used = ((double) (end - start)) / CLOCKS_PER_SEC;
            printf("線形探索の実行時間: %f 秒\n", cpu_time_used);
            printf("成功\n");
        }
    }
}
void binary_find(int id) {
    clock_t start, end;
    double cpu_time_used;
    int left = 0;
    int right = profile_data_nitems - 1;
    start = clock();

    while (left <= right) {
        int mid = left + (right - left) / 2;

     
        if (profile_data_store[mid].id == id) {
            end = clock();
            cpu_time_used = ((double) (end - start)) / CLOCKS_PER_SEC;
            printf("線形探索の実行時間: %f 秒\n", cpu_time_used);
            printf("終了\n");
            break;
        }
     
        else if (profile_data_store[mid].id < id) {
            left = mid + 1;
        }
 
        else {
            right = mid - 1;
        }
    }
}
\end{Verbatim}

%--------------------------------------------------------------------%
% 参考文献
%   以下は,書き方の例である.実際に,参考にした書籍等を見て書くこと.
%   本文で引用する際は,\cite{book:algodata}などとすればよい.
\begin{thebibliography}{99}
  \bibitem{www:label} C言語の構造体メンバのアライメント,https://memo.yammer.jp/posts/willani-struct-alignment,7/22/2024アクセス.
\end{thebibliography}

%--------------------------------------------------------------------%
%% 本文はここより上に書く(\begin{document}~\end{document}が本文である)
\end{document}
